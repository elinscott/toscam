%% LyX 1.6.7 created this file.  For more info, see http://www.lyx.org/.
%% Do not edit unless you really know what you are doing.
\documentclass[11pt,english]{article}
\usepackage[T1]{fontenc}
\usepackage[latin9]{inputenc}
\usepackage[a4paper]{geometry}
\geometry{verbose,lmargin=3cm,rmargin=3cm}
\usepackage{amsmath}
\usepackage{amssymb}
\usepackage{esint}

\makeatletter
%%%%%%%%%%%%%%%%%%%%%%%%%%%%%% User specified LaTeX commands.


\usepackage{latexsym}\usepackage{amsthm}\usepackage{bm}

\makeatother

\usepackage{babel}

\begin{document}

\title{Using the Pseudoatomic Solver to Generate NGWFs in ONETEP}


\author{Nicholas D.M. Hine \\
 Imperial College London}


\date{September 2011}

\maketitle

\section*{What is being calculated?}

When the atomic solver is used, a Kohn-Sham DFT calculation is performed
for a {}``pseudoatom''. This means that the pseudopotential of a
single isolated ion is used as the external potential, and the single-electron
Kohn-Sham states are solved self-consistently, for a given XC functional.
The resulting states can be expected to form an `ideal' atomic orbital
basis for a given calculation. In practice they are a much better
choice for molecular or crystalline systems than, say, the all-electron
orbitals of the STO-3G set (which is what you get if you leave the
entries in the species\_atomic\_set block set to ``AUTO''). They therefore
make rather good initial NGWFs, or a passable fixed basis for calculations
without NGWF optimisation, and should be at least comparable to the
equivalent basis sets generated in SIESTA, for example

The pseudoatomic orbitals we are looking for solve the Kohn-Sham equation:\[
\hat{H}_{\mathrm{KS}}\psi_{n}(\mathbf{r})=\epsilon_{n}\psi_{n}(\mathbf{r})\]
where\[
\hat{H}_{\mathrm{KS}}=-\frac{1}{2}\nabla^{2}+V_{\mathrm{loc}}(r)+\sum_{i}|p_{i}\rangle D_{ij}\langle p_{j}|\]
is the Hamiltonian in terms of kinetic, local potential and nonlocal
potential contributions. For a norm conserving pseudopotential the
matrix $D_{ij}$ is diagonal and there is one term per angular momentum
channel, so there is generally only one contributing nonlocal projector
for each wavefunction. In PAW and USPs, there may be more, and there
maybe cross-terms $D_{ij}$ for $i\neq j$. Because it is a spherical
potential, this form of the Kohn-Sham equation is easily separable
into an angular part, solved by the spherical harmonics ($Y_{lm}(\hat{\mathbf{r}})$,
or equivalently the real spherical harmonics $S_{lm}(\hat{\mathbf{r}})$)
multiplied by radial part, which we will be solving explicitly in
terms of a basis. 

The atomic orbitals are solved in a sphere of radius $R_{c}$. The
most appropriate basis to use therefore consists of normalised spherical
Bessel functions of given angular momentum $l$. The radial parts
of the basis functions, $B_{l,\nu}(r)$ are\[
B_{l,\nu}(r)=j_{l}(q_{l,\nu}r)\:/\,\Big[\int_{0}^{R_{c}}|j_{l}(q_{l,\nu}r)|^{2}r^{2}dr\Big]^{\frac{1}{2}}\;,\]
with $q_{l,\nu}$ such that $q_{l,\nu}R_{c}$ are the zeros of the
spherical Bessel functions. Thence $B_{l,\nu}(R_{c})=0$ and $\int_{0}^{R_{c}}|B_{l,\nu}(r)|^{2}r^{2}dr=1$
for all $\nu$. The basis would be complete if $\nu$ were infinite:
in practice it must be truncated, and the number of functions included
is determined by a kinetic energy cutoff $E_{\mathrm{cut}}$. The
criterion $\frac{1}{2}q_{l,\nu}^{2}<E_{\mathrm{cut}}$ determines
the largest $\nu$ for each $l$.

In the pseudoatom calculation we are therefore calculating Kohn-Sham
states of the form

\[
\psi_{n}(\mathbf{r})=\sum_{\nu}c_{n,\nu}\, B_{l_{n},\nu}(r)\, S_{l_{n}m_{n}}(\hat{\mathbf{r}})\;,\]
which have eigenvalues $\epsilon_{n}$ and occupancies $f_{n}$ which
include spin-degeneracy. The occupancies are fixed, and determined
outside of the main calculation, such that they sum to the number
of valence electrons. Spherical symmetry of the density is assumed,
so the occupancies of all members of a given set of $m$-degenerate
orbitals are always equal --- and in fact in practice the $m$ states
for a given $l$, $n$ pair are amalgamated into one state with $f_{n}$
summed over all the degenerate $m$'s. For example, for a nitrogen
ion with valence configuration $2s^{2}\,2p^{3}$, we would have $f_{2s}=2$,
$f_{2p}=3$. For this, we therefore need to find the lowest-energy
self-consistent eigenstate of each of $l=0$ and $l=1$. Henceforth
we will only consider the radial dependence $\psi_{n}(r)$. All radial
quantities will be considered to have been integrated over solid angle
already, so factors of $4\pi$ are omitted and $\int|\psi_{n}(r)|^{2}r^{2}dr=1$
for a normalised orbital.

We define the local potential through\[
V_{\mathrm{loc}}(r)=V_{\mathrm{psloc}}(r)+V_{H}[n](r)+V_{XC}[n](r)+V_{\mathrm{conf}}(r)\]
where for a spherical charge distribution $n(r)=\sum_{n}f_{n}|\psi_{n}(r)|^{2}$,
the Hartree potential is given by\[
V_{H}(r)=\frac{1}{r}\int_{0}^{r}n(r')r'^{2}dr'+\int_{r}^{\infty}n(r')r'\, dr'\;.\]
and the XC potential is $V_{XC}[n](r)=\frac{\partial E_{XC}[n]}{\partial n(r)}$.
$V_{\mathrm{conf}}(r)$ is an optional confining potential whose specific
form will be discussed later.

For each $l$ we can define the Hamiltonian matrix\[
H_{\nu,\nu'}^{l}=\int_{0}^{R_{c}}B_{l,\nu}(r)\left[\hat{H}B_{l,\nu'}(r)\right]r^{2}dr\]
and the overlap matrix\[
S_{\nu,\nu'}^{l}=\int_{0}^{R_{c}}B_{l,\nu}(r)B_{l,\nu'}(r)r^{2}dr\]
We then solve the secular equation\[
\mathbf{H}^{l}.\mathbf{c}_{n}=\epsilon_{n}\mathbf{S}^{l}.\mathbf{c}_{n}\]
to give the coefficients $c_{n,\nu}$ which describe the orbitals.
The orbitals are generated on the real-space grid and density mixing
with a variable mixing parameter $\alpha$ is then used until self-consistency
is obtained. The result is deemed to be converged once a) the Harris-Foulkes
estimate of the total energy (the bandstructure energy) matches the
total energy as determined from the density to within a given tolerance
(10$^{-5}$ Ha) and the energy has stopped changing each iteration,
to within a given tolerance ($10^{-7}$ Ha). 


\section*{Performing a Calculation with the Pseudoatomic Solver}

To use the atomic solver in a calculation, simply add the string {}``SOLVE''
for the elements you would to initialise to the pseudoatomic orbitals,
in \texttt{\%block species\_atomic\_set}. The code will then attempt
to determine which orbitals to solve for, and what their default occupancies
should be. To illustrate what will happen, we present some simple
examples. 

Let us imagine setting up a calculation with only nitrogen atoms,
for which $Z_{\mathrm{atom}}=7$ and $Z_{\mathrm{ion}}=5$. We want
to use $4$ NGWFs of radius $R_{c}=8.0$ per atom, so we would have
the following blocks in our input file:

\texttt{}%
\begin{minipage}[t]{1\columnwidth}%
\texttt{\vskip0.0cm}

\texttt{\%block species\_atomic\_set}

\texttt{N N 7 4 8.0}

\texttt{\%endblock species\_atomic\_set}

\texttt{\vskip0.0cm}

\texttt{\%block species\_atomic\_set}

\texttt{N {}``SOLVE''}

\texttt{\%endblock species\_atomic\_set}

\texttt{\vskip0.2cm}%
\end{minipage}

Note that because we may well want to add extra options to this string
later, it's best to always use the {}``'' quotes around SOLVE. These
settings will activate the pseudoatomic solver and it will attempt
to guess a default configuration for the atom. Since $Z_{\mathrm{ion}}=5$,
the code will count back five electrons from the end of the default
neutral atom occupancy, which is $1s^{2}\,2s^{2}\,2p^{3}$, and will
discover that the valence states are $2s^{2}\,2p^{3}$. Since you
have asked for $N=4$ NGWFs, the solver will then count forward from
the start of the valence states and determine that by including the
whole first set of $s$ and $p$ states it has enough to span the
valence space and create four orbitals (and thus four NGWFs). The
solver will therefore solve for one state with $l=0$, $f=2$ and
one state with $l=1$, $f=3$, all with radius $R_{c}=8.0$, and from
these states will produce one $s$-like NGWF and the three degenerate
$p_{x}$, $p_{y}$ and $p_{z}$ NGWFs.

A slightly more complex example would be if we were generating orbitals
for iron ($Z_{\mathrm{atom}}=26$, $Z_{\mathrm{ion}}=8)$:

\texttt{}%
\begin{minipage}[t]{1\columnwidth}%
\texttt{\vskip0.0cm}

\texttt{\%block species}

\texttt{Fe Fe 26 9 10.0}

\texttt{\%endblock species}

\texttt{\vskip0.0cm}

\texttt{\%block species\_atomic\_set}

\texttt{Fe {}``SOLVE''}

\texttt{\%endblock species\_atomic\_set}

\texttt{\vskip0.2cm}%
\end{minipage}

This time, to find the default configuration, the solver initialisation
routines will count back 8 electrons from the neutral atom configuration
of $1s^{2}\,2s^{2}\,2p^{6}\,3s^{2}\,3p^{6}\,3d^{6}\,4s^{2}$ and thus
will determine that the valence states are $3d^{6}\,4s^{2}$. However,
this time we have asked for 9 NGWFs, so it will then count forward
from $3d$, include the fivefold-degenerate lowermost $d$-like state
and the lowest $s$-like state. This only makes six, so it then will
also have to include the threefold-degenerate $4p$-like state. The
solver will have to solve for one unoccupied $p$-like orbital, which
will have $f=0$ throughout the calculation.


\subsection*{Controlling the configuration}

The default neutral-atom configurations for all the elements up to
$Z=92$ are included in the code, and will be used by default to generate
the configuration. However, it is also possible to override these
default configurations. For example, to generate NGWFs for iron in
the 3+ state, we might want to set the occupancies to $3d^{5}\,4s^{0}$.
To do this we use the {}``conf='' directive after the SOLVE string:

\texttt{}%
\begin{minipage}[t]{1\columnwidth}%
\texttt{\vskip0.0cm}

\texttt{\%block species\_atomic\_set}

\texttt{Fe {}``SOLVE conf=3d5 4s0''}

\texttt{\%endblock species\_atomic\_set}

\texttt{\vskip0.2cm}%
\end{minipage}

Any terms in the configuration which are not overridden are left at
their default values. Another example might be if we wanted to force
the partial occupation of more higher-lying states than would otherwise
be occupied for the neutral atom:

\texttt{}%
\begin{minipage}[t]{1\columnwidth}%
\texttt{\vskip0.0cm}

\texttt{\%block species\_atomic\_set}

\texttt{C {}``SOLVE conf=2s1.5 2p2.5''}

\texttt{\%endblock species\_atomic\_set}

\texttt{\vskip0.2cm}%
\end{minipage}

Note that the solver counts through the configuration terms strictly
in the order $n,l$, i.e.~$n$ is looped over outermost, then $l=0$
to $l=n-1$ for each $n$ innermost. This means that sometimes a little
thought is required to get the terms one actually wants, and not spurious
extra ones. For example, if we wanted to run a calculation of oxygen
with 9 NGWFs per atom, what we probably wanted would be to run with
1 $s$-like NGWF, 3 $p$-like NGWFs and 5 $d$-like NGWFs. However,
this is not by default what one will get if one asks for 

\texttt{}%
\begin{minipage}[t]{1\columnwidth}%
\texttt{\vskip0.0cm}

\texttt{\%block species}

\texttt{O O 9 9 9.0}

\texttt{\%endblock species}

\texttt{\vskip0.0cm}

\texttt{\%block species\_atomic\_set}

\texttt{O {}``SOLVE''}

\texttt{\%endblock species\_atomic\_set}

\texttt{\vskip0.2cm}%
\end{minipage}

This will identify $2s^{2}\,2p^{4}$ as the valence orbitals, and
counting forward will identify $2s$, $2p$, $3s$, $3p$ and just
1 of the 5 degenerate $3d$ states as the NGWFs required. Therefore,
we must instruct the atomsolver to ignore the unwanted excited $3s$
and $3p$ terms. We do this with an {}``X'', which instructs the
solver to knock out this term:

\texttt{}%
\begin{minipage}[t]{1\columnwidth}%
\texttt{\vskip0.0cm}

\texttt{\%block species\_atomic\_set}

\texttt{O {}``SOLVE conf=2s2 2p4 3sX 3pX 3d0''}

\texttt{\%endblock species\_atomic\_set}

\texttt{\vskip0.2cm}%
\end{minipage}

Strictly speaking, the $2s$, $2p$ and $3d$ strings are not needed,
as they are the default values anyway, but they are left in for clarity.
I find it advisable, so that I can keep track of the terms which will
generate the NGWFs, to add explicitly the terms with zero occupancy
to the conf string.


\subsection*{Generating larger, non-minimal bases}

ONETEP is generally used to create an \emph{in-situ-optimised}, minimal
basis (eg 4 NGWFs/atom for C, N, O etc). However, it is also possible
to fix the NGWFs and run with a much larger, unoptimised basis, in
a manner akin to other DFT codes designed for large-scale simulations
(eg SIESTA). In general, one would then want to use multiple NGWFs
for each angular momentum channel. This is known as using a {}``multiple-zeta''
basis set, where zeta refers to the radial part of the valence atomic
orbitals. For example, a {}``triple-zeta'' basis for carbon would
have $3$ $s$-like functions and $3$ of each of $p_{x}$, $p_{y}$,
and $p_{z}$-like functions. There are two approaches to generating
these extra functions. This simplest is just generate the higher-lying
orbitals of a given angular momentum. For carbon, for example, a double-zeta
basis in this scheme would include $3s$ and $3p$-like states. This
approach, however, is not very quick to converge with basis size.

It is often better to apply the commonly-used {}``split-valence''
approach. This allows the orbitals that have been generated to be
``split'' into multiple functions, so as to generate so-called {}``split-valence
multiple-zeta'' basis sets. In this formalism, one function $f(r)$
can be split into two functions $g_{1}(r)$ and $g_{2}(r)$ according
to the following: 
\begin{enumerate}
\item A matching radius $r_{m}$ is chosen: for $r>r_{m}$, we set $g_{2}(r)=f(r)$.
For $r\leq r_{m}$, we set $g_{2}(r)=r^{l}(a_{l}-b_{l}r^{2})$, where
$a_{l}$ and $b_{l}$ are chosen such that $g_{2}(r_{m})=f(r_{m})$
and $g_{2}'(r_{m})=f'(r_{m})$.
\item The other function, $g_{1}(r)$, is set to $f(r)-g_{2}(r)$, so $g_{1}(r)=0$
for $r\geq r_{m}$.
\item Both functions are renormalised, so $\int_{0}^{R_{c}}|g_{1}(r)|^{2}r^{2}dr=1$
and $\int_{0}^{R_{c}}|g_{2}(r)|^{2}r^{2}dr=1$.
\end{enumerate}
Splitting of a term is activated by adding a colon after the term
and specifying the ``split norm'' value. This is the fraction $p$
of the total norm of the orbital which is beyond the matching radius
$r_{m}$, such that $\int_{r_{m}}^{R_{c}}|f(r)|^{2}r^{2}dr=p$. If
this colon is present, the solver will take into account the total
number of orbitals which will result from this term \emph{after splitting,
}when counting forward in the configuration terms to determine which
orbitals to solve. For example, if we wished to generate a Double-Zeta
Polarisation (DZP) basis for oxygen ($2\times1\times s$, $2\times3\times p$,$1\times5\times d$),
where the last 15\% of the norm was matched for the $s$ and $p$-orbitals,
we would use the following:

\texttt{}%
\begin{minipage}[t]{1\columnwidth}%
\texttt{\vskip0.0cm}

\texttt{\%block species}

\texttt{O O 9 13 9.0}

\texttt{\%endblock species}

\texttt{\vskip0.0cm}

\texttt{\%block species\_atomic\_set}

\texttt{O {}``SOLVE 2s2:0.15 2p4:0.15 3sX 3pX 3d0''}

\texttt{\%endblock species\_atomic\_set}

\texttt{\vskip0.2cm}%
\end{minipage}


\subsection*{Overriding radii}

By default, the cutoff radius used for all the orbitals of an atom
is the same $R_{c}$ as defined in the \texttt{\%block species} entry
for that element. However, we can override this, either for all orbitals,
or for certain angular momentum channels.

To override the radius for all channels, for example to 7.0$a_{0}$,
would we add the flag {}``R=7.0'' to the SOLVE string:

\texttt{}%
\begin{minipage}[t]{1\columnwidth}%
\texttt{\vskip0.0cm}

\texttt{\%block species\_atomic\_set}

\texttt{O {}``SOLVE 2s2:0.15 2p4:0.15 3sX 3pX 3d0 R=7.0''}

\texttt{\%endblock species\_atomic\_set}

\texttt{\vskip0.2cm}%
\end{minipage}

Or leave the default values for all other channels, but override the
$d$-channel only to 5.0$a_{0}$, we would use

\texttt{}%
\begin{minipage}[t]{1\columnwidth}%
\texttt{\vskip0.0cm}

\texttt{\%block species\_atomic\_set}

\texttt{O {}``SOLVE 2s2:0.15 2p4:0.15 3sX 3pX 3d0 Rd=5.0''}

\texttt{\%endblock species\_atomic\_set}

\texttt{\vskip0.2cm}%
\end{minipage}


\subsection*{Adjusting confining potentials}

By default, a confining potential is applied, of the form:\[
V_{\mathrm{conf}}(r)=S\,\exp[-w_{l}/(r-R_{c}+w_{l})]/(r-R_{c})\]
where $S$ is the maximum height of the confining potential (at $r=R_{c}$),
and $w_{l}$ is the width of the region over which it is applied.
By default, $S=100$ Ha, and$w_{l}=3.0a_{0}$ for all $l$-channels
used. These can also be overridden, either all at once or for specific
$l$-values in the case of $w$.

For example, to set no confining potential on the confined $d$-orbitals
in Zinc, but to keep the default one on all the other orbitals, we
could set $w_{d}=0$:

\texttt{}%
\begin{minipage}[t]{1\columnwidth}%
\texttt{\vskip0.0cm}

\texttt{\%block species\_atomic\_set}

\texttt{Zn {}``SOLVE 3d10 4s2 Rd=5.0 wd=0.0''}

\texttt{\%endblock species\_atomic\_set}

\texttt{\vskip0.2cm}%
\end{minipage}

Or to turn off confinement potentials entirely, and generate $R_{c}=15.0a_{0}$
orbitals to match those generated by CASTEP's atomsolver (this should
allow direct comparison of energies, given suitable tweaking of the
energy cutoffs so that they exactly match:

\texttt{}%
\begin{minipage}[t]{1\columnwidth}%
\texttt{\vskip0.0cm}

\texttt{\%block species\_atomic\_set}

\texttt{O {}``SOLVE R=15.0 S=0.0''}

\texttt{\%endblock species\_atomic\_set}

\texttt{\vskip0.2cm}%
\end{minipage}

Note that there can be problems with convergence for certain choices
of confining potential. In particular, if you apply different confining
potentials to different \emph{occupied} orbitals, there will be problems
obtaining agreement between the Harris-Foulkes estimator and the actual
total energy - because the band energy will incorporate the different
confining potentials, but the total energy cannot. The confining potential
on angular momentum channels with no occupied orbitals can therefore
be whatever you like, but those of occupied orbitals must all match.
The exception to this is if the cutoff radius ment of one channel
is less than the onset radius for the others. In this case, there
is no need to apply a confinement to the lower-cutoff channel at all
(eg in the example above for Zinc).
\end{document}
